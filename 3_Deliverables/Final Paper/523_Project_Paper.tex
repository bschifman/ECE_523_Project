\documentclass{article}

% if you need to pass options to natbib, use, e.g.:
% \PassOptionsToPackage{numbers, compress}{natbib}
% before loading nips_2016
%
% to avoid loading the natbib package, add option nonatbib:
% \usepackage[nonatbib]{nips_2016}

%\usepackage{nips_2016}

% to compile a camera-ready version, add the [final] option, e.g.:
% \usepackage[final]{nips_2016}

\usepackage[utf8]{inputenc} % allow utf-8 input
\usepackage[T1]{fontenc}    % use 8-bit T1 fonts
\usepackage{hyperref}       % hyperlinks
\usepackage{url}            % simple URL typesetting
\usepackage{booktabs}       % professional-quality tables
\usepackage{amsfonts}       % blackboard math symbols
\usepackage{nicefrac}       % compact symbols for 1/2, etc.
\usepackage{microtype}      % microtypography

\title{Money Making Shit, Dope, The Dope}
\usepackage[final]{nips_2016}

% The \author macro works with any number of authors. There are two
% commands used to separate the names and addresses of multiple
% authors: \And and \AND.
%
% Using \And between authors leaves it to LaTeX to determine where to
% break the lines. Using \AND forces a line break at that point. So,
% if LaTeX puts 3 of 4 authors names on the first line, and the last
% on the second line, try using \AND instead of \And before the third
% author name.

\author{
  Benjamin A. Schifman, Justin JJJJJJJJJJJ. Siekmann\\
  Department of Electrical and Computer Engineering\\
  University of Arizona\\
  Tucson, AZ 85719 \\
  \texttt{bschifman@email.arizona.edu} \\
  \texttt{jsiekmannemail.arizona.edu}
  %% examples of more authors
  %% \And
  %% Coauthor \\
  %% Affiliation \\
  %% Address \\
  %% \texttt{email} \\
  %% \AND
  %% Coauthor \\
  %% Affiliation \\
  %% Address \\
  %% \texttt{email} \\
  %% \And
  %% Coauthor \\
  %% Affiliation \\
  %% Address \\
  %% \texttt{email} \\
  %% \And
  %% Coauthor \\
  %% Affiliation \\
  %% Address \\
  %% \texttt{email} \\
}

\begin{document}
 

\maketitle

\begin{abstract}
  The abstract paragraph should be indented \nicefrac{1}{2}~inch
  (3~picas) on both the left- and right-hand margins. Use 10~point
  type, with a vertical spacing (leading) of 11~points.  The word
  \textbf{Abstract} must be centered, bold, and in point size 12. Two
  line spaces precede the abstract. The abstract must be limited to
  one paragraph.
\end{abstract}

\section{Introduction}
\indent In our project, we will explore different approaches to apply machine learning principles and algorithms to the financial world. There exist technical indicators traditionally used by analysts to evaluate and predict market and equity performance, as they “can provide a unique perspective on the strength and direction of the underlying price action”. Feature extraction could be used to determine relevant indicators while identifying irrelevant and redundant indicators. Different implementations of algorithms based on these indicators could be used to predict performance of individual equities, sectors, or overall markets. They could also be used to classify and identify the correlation and interdependencies between equities, sectors, and markets. Our goal is to implement these various approaches to determine their efficacy as enablers to financial analysis. The biggest obstacle we face is finding relevant  and ways to accurately test our implementations. This being said, below are extensive datasets on stock market pricing and volume data that will serve as the basis in generating technical indicator features to implement in our machine learning algorithms. From this project we hope to deepen our understanding of the usage cases for applying specific machine learning algorithms as well as expanding upon our technical analysis of the stock market and which indicators play a role in successful market analysis.

\section{Related Work}
\section{Methods/Approach}
\section{Results}
Words, Words, Worrdsdsdfdsfsdafsdf
\section{Conclusion}
It works?\\ Yeah. Yes. Yes, it works. That was a statement. Definitely not a question. <- Note the "period" (aka statement)

\section*{References}

References follow the acknowledgments. Use unnumbered first-level
heading for the references. Any choice of citation style is acceptable
as long as you are consistent. It is permissible to reduce the font
size to \verb+small+ (9 point) when listing the references. {\bf
  Remember that you can use a ninth page as long as it contains
  \emph{only} cited references.}
\medskip

\small

[1] Alexander, J.A.\ \& Mozer, M.C.\ (1995) Template-based algorithms
for connectionist rule extraction. In G.\ Tesauro, D.S.\ Touretzky and
T.K.\ Leen (eds.), {\it Advances in Neural Information Processing
  Systems 7}, pp.\ 609--616. Cambridge, MA: MIT Press.

[2] Bower, J.M.\ \& Beeman, D.\ (1995) {\it The Book of GENESIS:
  Exploring Realistic Neural Models with the GEneral NEural SImulation
  System.}  New York: TELOS/Springer--Verlag.

[3] Hasselmo, M.E., Schnell, E.\ \& Barkai, E.\ (1995) Dynamics of
learning and recall at excitatory recurrent synapses and cholinergic
modulation in rat hippocampal region CA3. {\it Journal of
  Neuroscience} {\bf 15}(7):5249-5262.

\end{document}